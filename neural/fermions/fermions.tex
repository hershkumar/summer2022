\documentclass{notes}

\usepackage{amsmath}
\usepackage{braket}
\usepackage{amssymb}
\usepackage{amsthm}
\usepackage{listings}
\usepackage{xcolor}
\usepackage{bm}
\usepackage{esint}
\usepackage{gensymb}
\usepackage{mathrsfs}

\definecolor{codegreen}{rgb}{0,0.6,0}
\definecolor{codegray}{rgb}{0.5,0.5,0.5}
\definecolor{codepurple}{rgb}{0.58,0,0.82}
\definecolor{backcolour}{rgb}{0.95,0.95,0.92}

\lstdefinestyle{mystyle}{
    backgroundcolor=\color{backcolour},   
    commentstyle=\color{codegreen},
    keywordstyle=\color{magenta},
    numberstyle=\tiny\color{codegray},
    stringstyle=\color{codepurple},
    basicstyle=\ttfamily\footnotesize,
    breakatwhitespace=false,         
    breaklines=true,                 
    captionpos=b,                    
    keepspaces=true,                 
    numbers=left,                    
    numbersep=5pt,                  
    showspaces=false,                
    showstringspaces=false,
    showtabs=false,                  
    tabsize=2
}

\lstset{style=mystyle}



\newtheorem{postulate}{Postulate}
\newtheorem{theorem}{Theorem}[section]
\newtheorem{corollary}{corollary}[section]
\newtheorem{lemma}[theorem]{Lemma}

\title{Fermionic NNVMC}
\date{}

\let\ve\varepsilon
\let\vec\bm
\begin{document}
\section*{Fermionic Ansatz}
We have a system of $N$ fermions, $N^\uparrow$ being spin up, and  $N^\downarrow$  being spin down:
\begin{align*}
	\underbrace{x_1,\dots, x_{N^\uparrow}}_{\text{Up spin}}, \underbrace{x_{N^\uparrow +1}, \dots, x_N}_{\text{Down spins}}
\end{align*}


We can construct  $\Phi^\uparrow$ and  $\Phi^\downarrow$:
 \begin{align*}
	 \Phi^\uparrow(x_1,\dots x_N)&= \frac{1}{\sqrt{N!}} \begin{vmatrix}
		 \phi_1(x_1, \overbrace{\underbrace{x_2,\dots x_{N^\uparrow}}_{\text{Symmetric}}, \dots x_N}^{\text{Remaining coordinates}}) & \phi_1(x_2, \underbrace{x_1,x_3,\dots x_{N^\uparrow}}_{\text{Symmetric}}, \dots x_N) & \cdots\\ 
		 \phi_2(x_1, \underbrace{x_2, \dots x_{N^\uparrow}}_{\text{Symmetric}}, \dots x_N) & \phi_2(x_2, \underbrace{x_1,x_3,\dots, x_{N^\uparrow}}_{\text{Symmetric}}, \dots x_N)& \\ 
		 \vdots & & \ddots 
	 \end{vmatrix}
\end{align*}
Where $\Phi^\downarrow$ is constructed in the analogous way, using $N^\downarrow$ neural network functions, $\phi_{N^\uparrow +
1}, \dots, \phi_{N}$.

Our ansatz is 
\begin{align*}
	\Psi(x_1,\dots x_N) &=  \underbrace{\Phi^\uparrow(x_1,\dots x_N)}_{\text{Anti. } \uparrow} \underbrace{\Phi^\downarrow(x_1,\dots
	x_N)}_{\text{Anti. } \downarrow} \underbrace{f(x_1,\dots x_N)}_{\text{Symmetric}}e^{-\sum x_i^2}
\end{align*}


\section*{Delta Function Sampling}
\subsection*{Energy Computation}
\begin{align*}
	E &= \frac{\int dx\, \psi^2 \left(-\frac{1}{2m}\frac{1}{\psi} \frac{d^2 \psi}{dx^2} + \frac{1}{2}m\omega^2 x^2\right)}{\int dx\, \psi^2} + \frac{\int dx\, N_\uparrow
	N_\downarrow g \psi^2 \left(x\right) \frac{\psi^2\left(x'\right)}{\psi^2 \left(x\right)} d \left(x_\mu\right)}{\int dx\, \psi^2 \left(x\right)} \\ 
	  &\to \braket{\underbrace{-\frac{1}{2m}\frac{1}{\psi} \frac{d ^2\psi}{d x^2} + \frac{1}{2}m\omega^2x^2}_{\text{E w/ no delta}} + \underbrace{N_\uparrow N_\downarrow g \frac{\psi^2
	  \left(x'\right)}{\psi^2 \left(x\right)}d \left(x_\mu\right)}_{\text{E w/ delta}}}
\end{align*}
Where $x_\mu$ is the coordinate of the first down-spin fermion.
\subsection*{Gradient Computation}

\begin{align*}
	\frac{\partial E}{\partial \theta} &= \frac{2 \int dx\, \psi^2 \left(x\right) \left(\frac{1}{\psi\left(x\right)}H\psi - E \right)\frac{1}{\psi} \partial_\theta
	\psi\left(x\right)}{\int dx\, \psi^2 \left(x\right)} \\ 
									   &= \frac{\int dx\, \psi^2 \left(x\right)
										   \left(\frac{2}{\psi}\left(-\frac{1}{2}\frac{d ^2\psi}{d x^2} + \frac{1}{2}m\omega^2 x^2 \psi + g \delta \left(x_1-x_\mu\right)\psi -
									   E\right)\frac{1}{\psi}\partial_\theta\psi\right)}{\int dx\, \psi^2 \left(x\right)} \\ 
									   &= \frac{\int dx\, \psi^2 \left(x\right) \left(\frac{2}{\psi\left(x\right)}\left(-\frac{1}{2}\frac{1}{\psi}\frac{d
									   ^2\psi}{dx^2} + \frac{1}{2}m\omega^2 x^2\right) - E\right)\partial_\theta \psi}{\int dx\, \psi^2\left(x\right)} +
									   \frac{\int dx\, \psi^2 g \delta\left(x_1-x_\mu\right) \frac{2}{\psi} \partial_\theta \psi}{\int dx\, \psi^2 \left(x\right)} \\ 
									   &= \frac{\int dx\, \psi^2 \left(x\right) \left(\frac{2}{\psi\left(x\right)}\left(-\frac{1}{2}\frac{1}{\psi}\frac{d
									   ^2\psi}{dx^2} + \frac{1}{2}m\omega^2 x^2\right) - E\right)\partial_\theta \psi}{\int dx\, \psi^2\left(x\right)} +
									   \frac{\int dx\, \psi^2 N_{\uparrow}N_\downarrow\frac{\psi^2 \left(x'\right)}{\psi^2 \left(x\right)}\frac{2g}{\psi
									   \left(x'\right)}\partial_\theta \psi\left(x'\right)d \left(x_\mu\right)}{\int dx\, \psi^2\left(x\right)} \\ 
									   &\to \braket{\left(-\frac{1}{2m}\frac{1}{\psi} \frac{d ^2\psi}{d x^2} + \frac{1}{2}m\omega^2x^2 -
									   E\right)\frac{2}{\psi}\partial_\theta \psi + \left(N_\uparrow N_\downarrow g \frac{\psi^2
	  \left(x'\right)}{\psi^2 \left(x\right)}d \left(x_\mu\right)\right)\frac{2\partial_\theta
\psi\left(x'\right)}{\psi\left(x'\right)}}
\end{align*}


\end{document}
